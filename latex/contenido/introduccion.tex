\section{Introducción}
  El mercado laboral está adaptandose poco a poco a las nuevas tecnologías, cada vez es más común utilizar el Internet para buscar trabajo o contratar gente. En los últimos años se ha vivido un auge de aplicaciones para solicitar servicios especializados, tales cómo limpieza y transporte. Los cuales, apoyados del alcance del Internet, han logrado expandir su visibilidad en el mercado.

  El Curriculum Vitae tradicional, consiste en una hoja de papel que se presenta al momento de solicitar un empleo. Hoy día es posible digitalizar ese documento con una presentación mucho más amigable tanto para su actualización como su distribución. Un archivo JSON, serviría de fuente de datos para actualizar la presentación del Curriculum. El ofertante de recurso humano, proporcionaria a un potencial empleador la liga correspondiente para consultar el CV del candidato y cotactarle desde la misma página.

  Linked es una opción viable a la publicación de un CV, sin embargo, estos servicios se hospedan en la nube y dependen de terceros. El Curriculum Vitae lleva consigo una relación estrecha entre quién lo otorga y quién lo recibe, algunos datos pudiesen ser reservados según el tipo de trabajo al que se esté presentando. Dichas razones parecen suficientes para justificar el que un profesionistas tenga siempre el control en contenido, diseño y difusión de su Curriculum Vitae.
