\documentclass[conference]{IEEEtran}
\IEEEoverridecommandlockouts
% The preceding line is only needed to identify funding in the first footnote. If that is unneeded, please comment it out.
\usepackage{cite}
\usepackage{amsmath,amssymb,amsfonts}
\usepackage{algorithmic}
\usepackage{graphicx}
\usepackage{textcomp}
\usepackage{xcolor}
\def\BibTeX{{\rm B\kern-.05em{\sc i\kern-.025em b}\kern-.08em
    T\kern-.1667em\lower.7ex\hbox{E}\kern-.125emX}}

\begin{document}

  \title{Código base para publicación del Curriculum Vitae en la Web}

  \author{
    \IEEEauthorblockN{Emir Herrera González}
    \IEEEauthorblockA{emir.herrera@gmail.com}
  }

  \maketitle

  \begin{abstract}
    El Curriculum Vitea es hoy dia la presentación por omisión de un profesionista al momento de ofertarse en el mercado de recursos humanos, i.e., pedir trabajo. El creciente uso del Internet ha transformado la dinámica social, superponiendo una capa virtual a nuestras relaciones económicas
    En esta plantilla se proporciona el código base para elaborar y publicar en GitHub, un Curriculum Vitae personalizado.

    El código base presentado, proporciona una forma fácil de generar un documento HTML en el cuál se detalla el CV de un profesionista, dicho CV posteriormente podrá imprimerse en PDF, enviarse por correo o publicarse en Github Pages, a modo de promocionar la oferta laboral.
  \end{abstract}

  \begin{IEEEkeywords}
    curriculum, vitae, cv, trabajo, empleo, rh, recurso, humano, react, reactjs, www
  \end{IEEEkeywords}

  \section{Introducción}
    El mercado laboral está adaptandose poco a poco a las nuevas tecnologías, cada vez es más común utilizar el Internet para buscar trabajo o contraatar gente. En los últimos años se ha vivido una auge de aplicaciones para solicitar servicios especializados, tales cómo limpieza y transporte. Los cuales, apoyados del alcance del Internet, han logrado expandir su visibilidad en el mercado.

    El Curriculum Vitae tradicional, consiste en una hoja de papel que una presenat al momento de solicitar un empleo. Hoy día es posible digitalizar ese documento con una presentación mucho más amigable tanto en su actualización como su distribución. Un JSON, serviría de fuente de datos para ir actualizar la presentación del Curriculum. El ofertatnet de recurso humano, proporcionaria a un potencial empleador la liga correspondiente para consultar el CV del candidato.

\end{document}
